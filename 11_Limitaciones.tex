La sección de ``Limitaciones y posibilidad de éxito'' en un proyecto de tesis tiene como propósito identificar los posibles obstáculos o restricciones que podrían afectar el desarrollo de la investigación, así como evaluar las condiciones que favorecen su viabilidad y éxito.

\textbf{¿Qué debe incluir esta sección?}

\begin{enumerate}
    \item Limitaciones
    
    Aquí se describen los factores que podrían representar retos o restricciones, tales como:

    \begin{itemize}
        \item Limitaciones técnicas: falta de herramientas, software o equipamiento especializado.
        \item Limitaciones metodológicas: dificultades para obtener ciertos datos, acceso restringido a muestras o sujetos de estudio.
        \item Limitaciones de tiempo: el periodo disponible para el desarrollo del proyecto.
        \item Limitaciones económicas: recursos financieros insuficientes.
        \item Limitaciones éticas o legales: barreras normativas o permisos requeridos para trabajar con cierta información o población.
    \end{itemize}

    Estas no son necesariamente fallas del proyecto, sino condiciones reales que se deben anticipar y considerar.

    \item Posibilidad de éxito
    
    Después de exponer las limitaciones, se debe presentar una reflexión sobre:
    \begin{itemize}
        \item La factibilidad técnica del proyecto, dados los recursos y conocimientos disponibles.
        \item El respaldo institucional, como asesoría académica o apoyo de laboratorios.
        \item La experiencia previa del estudiante o del equipo, que contribuya a resolver obstáculos.
        \item El impacto potencial de los resultados, que justifica el esfuerzo a pesar de las limitaciones.
    \end{itemize}

    Esta parte ofrece una visión equilibrada y realista de lo que se puede lograr.
\end{enumerate}


\textbf{Extensión sugerida}

Una cuartilla o menos, escrita con claridad y sin tecnicismos innecesarios.