El resumen es una sección breve y concisa que debe ofrecer una visión general del anteproyecto de tesis. Su propósito es comunicar, en media cuartilla, los elementos más relevantes del trabajo propuesto.

Debe incluir:

\begin{enumerate}
    \item Problema a resolver: Contextualiza brevemente la problemática que motiva el estudio.
    \item Objetivo general: Señala claramente el propósito central de la investigación.
    \item Metodología propuesta: Describe de manera general el enfoque y herramientas que se utilizarán.
    \item Resultados esperados: Menciona los productos o contribuciones que se prevé obtener.
    \item Relevancia: Justifica brevemente la importancia e impacto potencial del trabajo.
\end{enumerate}

\textbf{Extensión máxima}: media cuartilla (aproximadamente 150 a 200 palabras).

\textbf{Estilo}: académico, claro y directo, evitando detalles técnicos extensos.