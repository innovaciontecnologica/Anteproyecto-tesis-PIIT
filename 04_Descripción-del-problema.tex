La descripción del problema es el apartado donde se expone de manera clara, específica y fundamentada la situación problemática que se desea investigar. Es uno de los núcleos más importantes del trabajo, ya que orienta todo el desarrollo posterior de la tesis (objetivos, hipótesis, metodología, etc.).

Su propósito es identificar una necesidad, conflicto, vacío de conocimiento o situación no resuelta dentro de un contexto determinado, que justifique la realización del estudio.

\begin{enumerate}
    \item Contextualización del problema

    Se explica brevemente el entorno en el que ocurre el problema: sector industrial, área de conocimiento, población afectada, campo tecnológico, etc.

    \item Descripción clara y específica
    
    Aquí se detalla la problemática:

    \begin{itemize}
        \item ¿Qué está ocurriendo?
        \item ¿Por qué es un problema?
        \item ¿Quiénes son los afectados?
        \item ¿Qué consecuencias tiene?
    \end{itemize}
\end{enumerate}

\textbf{Recomendaciones de redacción}:

\begin{itemize}
    \item Utiliza lenguaje técnico y académico, pero accesible.
    \item Evita afirmaciones sin fundamento; sustenta con datos, fuentes o evidencia cuando sea posible.
    \item No incluyas aún los objetivos ni las soluciones propuestas (eso va en otras secciones).
\end{itemize}

\textbf{Extensión sugerida}:

De media a una cuartilla completa, dependiendo de la complejidad del problema y el nivel del posgrado (maestría o doctorado).
