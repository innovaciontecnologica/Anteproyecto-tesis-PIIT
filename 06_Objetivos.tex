Los objetivos definen de forma precisa lo que se pretende lograr con la investigación. Sirven como guía para el desarrollo del trabajo, delimitando el alcance y orientando las metodologías a emplear.

\textbf{Tipos de objetivos}

\begin{enumerate}
    \item Objetivo General
    \begin{itemize}
        \item Describe la finalidad principal del estudio.
        \item Resume el propósito central de manera clara y directa.
        \item Suele redactarse con un verbo en infinitivo: diseñar, desarrollar, analizar, implementar, evaluar, etc.
    \end{itemize}
    \item Objetivos Específicos
    \begin{itemize}
        \item Derivan del objetivo general.
        \item Son metas parciales que, al cumplirse, permiten alcanzar el objetivo general.
        \item Deben ser concretos, medibles, realistas y ordenados lógicamente.
    \end{itemize}

\end{enumerate}

\textbf{Recomendaciones para redactar los objetivos}

\begin{itemize}
    \item Utiliza verbos que expresen acciones observables o evaluables. Evita verbos vagos como ``comprender'' o ``estudiar''.
    \item Asegúrate de que los objetivos específicos cubren todas las etapas del desarrollo de la tesis.
    \item Los objetivos deben ser coherentes con el problema planteado y con la justificación.
\end{itemize}