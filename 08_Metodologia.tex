La metodología es la sección del proyecto donde se describe cómo se llevará a cabo la investigación. Explica el enfoque, técnicas, herramientas, procedimientos y criterios que se utilizarán para recolectar, analizar e interpretar los datos que permitirán alcanzar los objetivos planteados.

Es una parte clave, pues garantiza la validez, replicabilidad y rigor científico del trabajo.

\textbf{¿Qué debe incluir la sección de Metodología?}

\begin{enumerate}
    \item Tipo de investigación
    \begin{itemize}
        \item Exploratoria, descriptiva, explicativa, aplicada, experimental, cualitativa, cuantitativa, mixta, etc.
    \end{itemize}
    \item Diseño metodológico
    \begin{itemize}
        \item Cómo se estructura la investigación: fases o actividades.
    \end{itemize}
    \item Población y muestra (si aplica)
    \begin{itemize}
        \item Descripción del universo de estudio.
        \item Criterios de selección de muestra o escenarios de prueba (usuarios, datos, dispositivos, etc.).
    \end{itemize}
    \item Instrumentos y técnicas de recolección de datos
    \begin{itemize}
        \item Encuestas, entrevistas, sensores, simuladores, plataformas digitales, bases de datos, etc.
        \item Justificación del uso de cada instrumento.
    \end{itemize}
    \item Técnicas de análisis
    \begin{itemize}
        \item Métodos estadísticos, herramientas computacionales, lenguajes de programación, software especializado, algoritmos de IA, etc.
        \item Explica cómo se procesarán los datos obtenidos.
    \end{itemize}
    \item Recursos requeridos
    \begin{itemize}
        \item Equipamiento, materiales, software, licencias, laboratorios, entre otros.
    \end{itemize}
    \item Consideraciones éticas
    \begin{itemize}
        \item Si la investigación involucra personas, datos sensibles o implica riesgos, debe incluirse cómo se protegerán los derechos de los participantes.
    \end{itemize}
\end{enumerate}

\textbf{Extensión sugerida}

Para un anteproyecto, la sección de metodología puede tener de 2 a 5 cuartillas, según el detalle requerido.

\textbf{Consejos}

\begin{itemize}
    \item Sé claro, preciso y evita ambigüedades.
    \item Justifica tus elecciones metodológicas.
    \item Relaciona la metodología con tus objetivos específicos.
\end{itemize}