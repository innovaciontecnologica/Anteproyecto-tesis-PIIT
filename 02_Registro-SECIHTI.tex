\section{Protocolo}

Un protocolo debe de ser breve y presentar la idea general del proyecto de tesis. Resume el problema a investigar, el objetivo principal, la metodología y la relevancia del tema.

Debe escribirse en una sola oración clara y concisa, con una extensión de entre 50 y 500 caracteres (incluyendo espacios).

\section{Impacto científico}

El impacto científico describe cómo los resultados del proyecto de tesis podrían contribuir al conocimiento en un área específica, ya sea al generar nuevas metodologías, validar teorías o resolver problemas relevantes.

Debe redactarse en una o dos frases claras, con una extensión entre 50 y 500 caracteres (incluyendo espacios).

\section{Impacto social}

El impacto social se refiere a los beneficios potenciales que el proyecto de tesis puede generar en la sociedad, como mejorar la calidad de vida, resolver problemáticas sociales, económicas o ambientales, o incidir en políticas públicas.

Debe redactarse en una o dos frases claras, con una extensión entre 50 y 500 caracteres (incluyendo espacios).

\section{Aportaciones a la solución de problemas prioritarios}

Las aportaciones a la solución de problemas prioritarios describen cómo el proyecto contribuye a resolver desafíos identificados como relevantes a nivel regional o nacional, especialmente aquellos alineados con los ejes del PRONACES del SECIHTI (como salud, medio ambiente, seguridad, tecnología, etc.).

Debe tener entre 50 y 500 caracteres (incluyendo espacios).

\section{Estrategias para el acceso universal al conocimiento}

Las estrategias para el acceso universal al conocimiento se refieren a las acciones contempladas en el proyecto para asegurar que los resultados, herramientas, tecnologías o saberes generados estén disponibles y sean accesibles para la sociedad, comunidades académicas o sectores interesados.

Debe tener entre 50 y 500 caracteres (con espacios).