El marco teórico es una fundamentación conceptual y científica que sustenta tu investigación. Se construye a partir del análisis y la integración de teorías, conceptos clave, antecedentes investigativos y enfoques metodológicos previos relacionados con tu tema. Sirve para:

\begin{itemize}
    \item Contextualizar el problema de investigación.
    \item Definir conceptos fundamentales.
    \item Justificar la pertinencia del estudio.
    \item Evitar duplicidad y orientar el desarrollo metodológico.
\end{itemize}

\textbf{¿Qué debe incluir?}

\begin{enumerate}
    \item Revisión de la literatura científica
    \begin{itemize}
        \item Trabajos relevantes publicados en revistas, libros o congresos.
        \item Estudios previos similares al tuyo, resaltando avances y vacíos.
    \end{itemize}
    \item Teorías y modelos aplicables
    \begin{itemize}
        \item Teorías que sustentan tu propuesta (por ejemplo: redes neuronales, modelado estadístico, teoría de sistemas, etc.).
        \item Marco conceptual que defina cómo entiendes y abordas tu tema.
    \end{itemize}
    \item Definición de conceptos clave
    \begin{itemize}
        \item Explica términos técnicos y específicos del área.
        \item Clarifica el uso de cada concepto dentro de tu enfoque.
    \end{itemize}
    \item Relación con tu investigación
    \begin{itemize}
        \item Señala cómo cada teoría, estudio o modelo contribuye a tu diseño metodológico, a la construcción de objetivos y a la validación de tu hipótesis.
    \end{itemize}
\end{enumerate}

\textbf{¿Cuánto debe extenderse?}

Para un anteproyecto de tesis, se recomienda que el marco teórico tenga de 5 a 10 cuartillas, aunque puede variar según la complejidad del tema. Debe ser:

\begin{itemize}
    \item Claro y estructurado.
    \item Con fuentes actuales y confiables (preferentemente últimos 5 años).
    \item Redactado en un lenguaje técnico pero accesible.
\end{itemize}

\textbf{Recomendaciones}

\begin{itemize}
    \item Organiza los temas por subapartados temáticos o cronológicos.
    \item Finaliza con una síntesis crítica que conduzca a tu propuesta de investigación.
\end{itemize}

Las citas se van a agregar de forma automática al momento de mencionarse en el texto, la referencia debe de estar en formato bibtex en el archivo de referencias \cite{Ellinidou2019} \cite{Benini2002,Ellinidou2019} \cite{Benini2002,dally2004principles,Ellinidou2019}.