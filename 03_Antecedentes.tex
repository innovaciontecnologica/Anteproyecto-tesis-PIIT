La sección de antecedentes es uno de los apartados más importantes del anteproyecto o protocolo de tesis. Su función principal es justificar la relevancia del estudio a través de una revisión crítica del conocimiento existente sobre el tema. Aquí se presentan los principales hallazgos, teorías, enfoques y tecnologías relacionados con la propuesta de investigación, de manera que el lector pueda entender de dónde surge la idea del trabajo, qué se ha hecho previamente y qué falta por hacer.

\textbf{Contenido sugerido}:

\begin{enumerate}
    \item Contextualización del problema
    Se comienza explicando el entorno general en el que se inscribe el problema: ¿qué se sabe hasta ahora?, ¿por qué es importante este tema?, ¿qué impacto tiene en el área académica, tecnológica o social?

    \item Revisión de literatura

    Aquí se citan y analizan los trabajos más relevantes relacionados con el tema: investigaciones científicas, artículos académicos, tesis previas, patentes, normas técnicas, entre otros. No se trata de enlistar fuentes, sino de integrar la información destacando:

    \begin{itemize}
        \item Qué han encontrado otros autores
        \item Qué métodos han utilizado
        \item Qué resultados han obtenido
        \item Qué vacíos o limitaciones persisten
    \end{itemize}

    \item Situación actual del conocimiento

    Se debe mostrar claramente cuál es el estado del arte. ¿Existen controversias? ¿Faltan datos? ¿Hay una necesidad específica que no ha sido resuelta? Esto ayuda a construir la justificación del proyecto.
\end{enumerate}

\textbf{Consideraciones de forma}:

\begin{itemize}
    \item Extensión: No debe exceder tres cuartillas, escritas en fuente legible (por ejemplo, Times New Roman 12 o Arial 11), interlineado 1.5 o doble.
    \item Redacción académica: Utilizar un lenguaje claro, técnico y objetivo.
    \item Cohesión: Evitar simplemente copiar fragmentos de textos. Se espera que haya análisis, síntesis y conexión lógica entre ideas.
\end{itemize}

Esta sección le permite al comité o lector identificar si el estudiante ha realizado una investigación preliminar seria y si su propuesta tiene pertinencia, originalidad y viabilidad dentro del marco del conocimiento existente.