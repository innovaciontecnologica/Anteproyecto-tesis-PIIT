\documentclass[10pt,letterpaper]{report}
\usepackage{lipsum} 
\usepackage[utf8]{inputenc}
\usepackage[T1]{fontenc}
\usepackage[spanish,mexico]{babel}
\usepackage{graphicx}
\usepackage[top=1.85in,bottom=1.7in,right=.75in,left=.75in,headheight=70pt,headsep=1cm,,footskip=90pt]{geometry}
\usepackage{chngcntr}
\counterwithout{table}{chapter}
\counterwithout{figure}{chapter}
%uso de tablas multicolumnas y color
\usepackage{multicol}
\usepackage{booktabs}
\usepackage{colortbl}
\usepackage{lscape}

%Uso de Referencias con bibtex

\usepackage[backend=bibtex]{biblatex}
\addbibresource{referencias.bib}


\begin{document}
	% Haciendo la portada
	\begin{titlepage}
		\centering
		\includegraphics[width=.7\textwidth]{Imagenes/Logo.png}\par\vspace{1cm}
		{\scshape\LARGE \bfseries Universidad Autónoma de Zacatecas \par 
        ``Francisco García Salinas''\par }
		\vspace{1cm}
		%{\scshape\Large Doctorado en Ingeniería para la Innovación Tecnológica\par}
        {\scshape\Large Maestría en Ingeniería para la Innovación Tecnológica\par}
		\vspace{1.5cm}
		{\huge\bfseries Título de anteproyecto de tesis\par}
		\vspace{2cm}
		{\Large\itshape Nombre del Alumno\par}
		\vfill
		Director de tesis :\par
		Asesor 1\par
				
		\vfill
		
		% Bottom of the page
		{\large Zacatecas, Zac. a \today\par}
	\end{titlepage}

	% Resumen
\chapter*{Resumen}
Debe dar una idea clara sobre el problema a investigar, su importancia y las necesidades de ser investigado. Deberá dejar en claro los objetivos de la investigación. Asimismo, el resumen debe contener los métodos y procedimientos contenidos en el capítulo de metodología.          

\tableofcontents
	
\chapter{Antecedentes}
Incluir los antecedentes del tema, enfatizando, en lo posible, resultados de proyectos locales, nacionales y/o internacionales relacionados al tema a investigar. 
\chapter{Descripción del problema a investigar}
Describir el problema que genera la investigación, evaluando críticamente su magnitud, efectos y causas existentes.

\chapter{Justificación y relevancia}
Describir la importancia y relevancia de la investigación y el efecto que el proyecto tendrá en mejorar los conceptos, métodos, tecnologías, tratamientos y servicios relacionados al tema de investigación.
\chapter{Impacto científico}
\input{05_Impacto-cientifico}
\chapter{Impacto social}
\input{06_Impacto-social}
\chapter{Aportaciones a la solución de problemas prioritarios}
\input{07_Aportaciones}
\chapter{Estrategias para el acceso universal del conocimiento}
\input{08_Estrategias}
\chapter{Objetivos}
\input{09_Objetivos}
\chapter{Marco Teórico}
\input{10_Marco-Teorico}
\chapter{Metodología}
\input{11_Metodologia}
\chapter{Descripción de los procedimientos a realizar}
Describir los procedimientos necesarios que aseguren las actividades relacionadas con el proyecto (pruebas, análisis, entrevistas, exámenes de laboratorio, procedimientos, etc.), y en consecuencia el éxito del mismo.
\chapter{Cronograma de actividades}
\input{13_Cronograma-de-actividades}
\chapter{Limitaciones y factibilidad de la investigación}
\input{14_Limitaciones}
\chapter{Comité Propuesto}
\input{15_Comite}
\chapter*{Referencias}
\printbibliography[title={Referencias}]




\end{document}