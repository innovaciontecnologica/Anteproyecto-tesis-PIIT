La descripción de los procedimientos detalla paso a paso cómo se llevará a cabo la investigación, desde la preparación inicial hasta la obtención y análisis de los resultados. Es la guía operativa que permite al lector entender qué se hará, en qué orden y con qué recursos, garantizando la reproducibilidad del estudio.

\textbf{¿Qué debe incluir esta sección?}

\begin{enumerate}
    \item Secuencia cronológica de actividades
    \begin{itemize}
        \item Describe de manera ordenada las fases de trabajo: planeación, desarrollo, pruebas, validación, análisis, etc.
    \end{itemize}
    \item Tareas específicas dentro de cada fase
    \begin{itemize}
        \item Qué actividades se realizarán en cada etapa (por ejemplo: diseño de circuitos, entrenamiento de modelos, recolección de datos, simulaciones, etc.).
    \end{itemize}
    \item Herramientas y materiales a utilizar
    \begin{itemize}
        \item Hardware, software, equipos de laboratorio, sensores, plataformas digitales, entre otros.
    \end{itemize}
    \item Duración estimada de cada actividad
    \begin{itemize}
        \item Especifica el tiempo previsto para cada procedimiento, idealmente complementado con un cronograma.
    \end{itemize}
    \item Responsables (si aplica)
    \begin{itemize}
        \item En proyectos colaborativos, puede indicarse quién se encargará de cada procedimiento.
    \end{itemize}
    \item Criterios de validación y seguimiento
    \begin{itemize}
        \item Indica cómo se evaluará el éxito de cada procedimiento y si habrá retroalimentación o ajustes en el proceso.
    \end{itemize}
\end{enumerate}

\textbf{Extensión sugerida}

De media a una cuartilla es suficiente para esta sección en un anteproyecto, siempre que sea clara, ordenada y detallada.