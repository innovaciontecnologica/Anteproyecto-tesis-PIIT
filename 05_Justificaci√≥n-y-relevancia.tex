La sección Justificación y Relevancia tiene como objetivo explicar por qué es importante realizar la investigación propuesta, tanto desde el punto de vista académico-científico como desde su posible impacto en la sociedad, el sector productivo o el desarrollo tecnológico.

Debe convencer al lector de que el problema planteado merece ser estudiado y que la solución o contribución esperada tendrá valor en el contexto actual.

\textbf{¿Qué debe contener?}

\begin{enumerate}
    \item Relevancia científica o académica
    \begin{itemize}
        \item ¿Cómo contribuirá la investigación al conocimiento en el área?
        \item ¿Rellena un vacío teórico, metodológico o experimental?
        \item ¿Se alinea con alguna línea de investigación institucional o del programa de posgrado?
    \end{itemize}
    \item Pertinencia tecnológica o aplicada
    \begin{itemize}
        \item ¿Ayuda a resolver un problema técnico o de innovación?
        \item ¿Tiene potencial para aplicarse en un contexto real?
    \end{itemize}
    \item Impacto social o económico
    \begin{itemize}
        \item ¿Cómo beneficiará a una comunidad, institución, sector o región?
        \item ¿Está alineada con alguna prioridad nacional o regional (por ejemplo, PRONACES, SECIHTI)?
    \end{itemize}
    \item Vinculación institucional o estratégica
    \begin{itemize}
        \item ¿Contribuye al fortalecimiento de redes de colaboración, cuerpos académicos o laboratorios?
    \end{itemize}
\end{enumerate}

\textbf{Estilo de redacción}

\begin{itemize}
    \item La redacción debe ser clara, argumentativa y fundamentada.
    \item No es una lista de beneficios, sino una reflexión crítica sobre la necesidad y utilidad del estudio.
    \item Puedes mencionar fuentes, datos estadísticos o referencias normativas para respaldar la importancia del tema.
\end{itemize}

\textbf{Extensión sugerida}

Entre media y una cuartilla, dependiendo del nivel del trabajo y de la complejidad del tema.